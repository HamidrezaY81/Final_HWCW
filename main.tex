\documentclass{article}

\usepackage{fancyhdr}
\usepackage{hyperref} 
\usepackage{geometry} 
\geometry{a4paper, margin=1in}
\usepackage{graphicx}

\title{Computer Workshop \\ Final Homework}
\author{Hamidreza Yadegari}
\date{winter of 4031}

\hypersetup{hidelinks}
\pagestyle{fancy}
\fancyhead[L]{\thepage}

\begin{document}

\maketitle 
\thispagestyle{empty}
\newpage

\tableofcontents
\fancyhead[R]{Final HomeWork}
\setcounter{page}{1}
\newpage

\section{Git and GitHub}
    \subsection{Repository Initialization and Commits}
        First, we create a repository in GitHub by entering the desired name in its creation and ticking the box to create a `README` file. Then, using the command `git clone <repository-SSH-link>`, we transfer the above repository to our system. After that, we create a file called `main.tex` to write the answers to the questions. 

        Next, we use the command `git add .` to add it to the staging area. Then, we commit the file with the command:
        \begin{verbatim}
            git commit -m "add main.tex"
        \end{verbatim}
            Finally, we apply the changes to GitHub with the command:
        \begin{verbatim}
            git push origin master
        \end{verbatim}
        Now the project is ready for the next stage.
        \\
        \subsection{GitHub Actions for LaTeX Compilation}
            To set up GitHub Actions for LaTeX compilation, we first create a directory named `.github` in the main directory of the remote repository. Inside `.github`, we create another directory named `workflows`. Then, we create a file called `latex-build.yml` inside the `workflows` directory and add the settings from the site:
            \url{https://mrturkmen.com/posts/build-release-latex/}.
            
            In the `root\_file`, `asset\_path`, and `asset\_name` fields of the YAML file, we replace the word `main` with the name of the LaTeX file created in the previous step. 
            
            After making these changes, we use the following commands to add and commit the changes:
            \begin{verbatim}
            git add .
            git commit -m "add .github"
            \end{verbatim}
            Finally, we push the changes to GitHub using:
            \begin{verbatim}
            git push origin main
            \end{verbatim}
            
\end{document}
